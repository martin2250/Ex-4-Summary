\section{Kernspin}\label{sec:kernspin}
Der Kernspin $I$ ist die Summe der Bahndrehimpulse und Spins der Nukleonen.
Bei Kernen mit mehreren Nukleonen bilden je zwei Protonen/Neutronen ein Paar mit Gesamtdrehimpuls $\vec{I} = 0$.\\
Daher gilt für den Kernspin $\vec{I} = \sum (\vec{s_i} + \vec{l_i})$ eines Kerns mit Nukleonenzahl $A$:
\begin{center}
\bgroup
\def\arraystretch{1.25}
\begin{tabular}{lll}
	$A$ gerade&	gg&	$I = 0$\\
	&	uu&	$I = 0,1,\dots$ (zwei ungepaarte $\tfrac{1}{2}$-Spins)\\
	$A$ ungerade&	gu/ug&	$I = \tfrac{1}{2},\tfrac{3}{2},\dots$ (ein ungepaarter $\tfrac{1}{2}$-Spins)\\
\end{tabular}
\egroup
\end{center}
, wobei gg/ug/uu je für eine gerade/ungerade Anzahl an Protonen und Neutronen steht.
