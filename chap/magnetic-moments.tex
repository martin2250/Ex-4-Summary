\section{Magnetisches Moment}
Ein Drehimpuls erzeugt immer ein magnetisches Moment.\\
Für den Bahndrehimpuls eines Elektrons gilt
\begin{equation*}
\vec{\mu_l}
= -g_l \cdot \mu_\text{B} \cdot \frac{\vec{l}}{\hbar}
= - \mu_\text{B} \cdot \frac{\vec{l}}{\hbar} \qquad (g_l = 1),
\end{equation*}
für den Spin eines Teilchens x gilt:
\begin{equation*}
\vec{\mu_s} = g_\text{x} \cdot \mu_\text{x} \cdot \frac{\vec{s}}{\hbar}
\end{equation*}
mit $\mu_\text{x} = \frac{\text{e}}{2 m_\text{x}}$ und den g-Faktoren
\begin{center}
\begin{tabular}{ll}
	Elektron&	$g_\text{e} \approx \num{-2.002}$\\
	Neutron&	$g_\text{n} \approx \num{-3.826}$\\
	Proton&	$g_\text{p} \approx \num{+5.586}$\\
\end{tabular}
\end{center}
g-Faktoren von ganzen Atomkernen werden experimentell bestimmt, dabei wird $\vec{s}$ durch $\vec{I}$ ersetzt.
