\section{Externen Felder}
\subsection{Drehimpulspräzession}
Bei vorhandener Quantisierungsachse (oBdA z-Achse) präzedieren Drehimpulse um ihre z-Komponente bzw bei zusammengesetzten Drehimpulsen, zB. $\vec{j} = \vec{l} + \vec{s}$ präzedieren die einzelnen Drehimpulse um den Gesamtdrehimpuls, dieser präzediert um seine z-Komponente.
Die Präzessionsfrequenz ist gegeben durch die Lamourfrequenz
\begin{equation*}
	\omega_\text{L} = \frac{\mu}{\hbar} \cdot g \cdot B_\text{ext}
\end{equation*}

\subsection{Zeemann Effekt}
Der Drehimpuls der Hülle wechselwirkt mit einem externen Magnetfeld
\begin{equation*}
	V_\text{Ze} = - \, \vec{\mu} \cdot \vec{B}_\text{ext} = - \, \mu \, m_J \cdot B_\text{ext}.
\end{equation*}

Der \textbf{normaler Zeemann Effekt} tritt auf wenn kein Gesamtspin vorliegt, jedes Energieniveau $L$ spaltet auf in $2L + 1$ ($L = \sum l$) Äquidistante Niveaus mit $\Delta E = \hbar \omega_\text{L} = (g_L \cdot) \mu_L B_\text{ext}$, unabh. von L.

Beim \textbf{anomalen Zeemann Effekt} tritt zusätzlich Spinmagnetismus auf. Es erfolgt eine Aufspaltung des $J$-Niveau in $2J + 1$ Niveaus, die wegen den unterschiedlichen $g$-Faktoren von Spin und Bahndrehimpuls abhängig von $J$ sind.

\subsection{Kernspinresonanz}

Äquivalent zum Zeemann Effekt, mit Kernmagneton:
\begin{align*}
	V_\text{pot} &= \underbrace{g_l \, \mu_N \, B_\text{ext}}_{|\Delta E|} \cdot m_I\\
	\rightarrow \omega &= g_l \, \mu_N \, B_\text{ext} = \omega_\text{L}
\end{align*}
Die Übergänge zwischen den neuen Energieniveaus können nur mit der Resonanzfrequenz $\omega$ angeregt werden.

\subsection{Stark-Effekt}

Wechselwirkung des elektrischen Dipols eines Atoms mit einem externen Magnetfeld, Energieshift proportional zu $E$ und $|m_J|$.
Es kann auch ein Dipolmoment induziert werden (Polarisation des Atoms) $\rightarrow \; \propto E^2$.
