\section{Energieniveaus H-Atom}
\begin{center}
	\includegraphics[width=.8\linewidth]{img/energieniveaus}\\
	(Energie nicht maßstabsgetreu)
\end{center}

\subsection{Bohr}
\begin{equation*}
	E_n = -hc R_\infty \frac{1}{n^2}
\end{equation*}
$R_\infty$ proportional zu $m_\text{e}$, ersetze $m_\text{e}$ durch reduzierte Masse oder zB. Myonenmasse.

\subsection{Dirac (Spin-Bahn-Kopplung)}
Wechselwirkung zwischen Spin- und Bahnmagnetismus:
\begin{equation*}
	V_{ls} = -\vec{\mu_s} \cdot \vec{B_l}
\end{equation*}
Aus $\vec{j} = \vec{l} + \vec{s}$ und Quantisierung $|\vec{j}| = \hbar \sqrt{j(j+1)}$ etc folgt
\begin{equation*}
	\vec{l} \cdot \vec{s} = \hbar^2 \frac{1}{2}\left(j(j+1)-l(l+1)-s(s+1)\right)
\end{equation*}
Ergibt (incl. relativistischer Effekte)
\begin{equation*}
	E_{nj} = E_n \left(1 + \frac{Z^2 \alpha^2}{n^2} \cdot \left(\frac{n}{j+\frac{1}{2}} - \frac{3}{4} \right)\right)
\end{equation*}
und liefert stets negative Verschiebung (da $E_n$ negativ), also eine stärkere Bindung.

\subsection{Lamb}
Wechselwirkung mit virtuellen Elektron-Positron Paaren heben Entartung von Zuständen mit gleichem j auf.

\subsection{Hyperfeinstruktur}
Wechselwirkung Kernspin - Bahndrehimpuls:
\begin{equation*}
	V_\text{HFS} = g_I \, \mu_\text{N} \, B_J \, \frac{\left[F(F+1)-I(I+1)-L(L+1)\right]}{2 \sqrt{J(J+1)}}
\end{equation*}
mit Bahnmagnetismus $B_J$.
Genaue Vorfaktoren werden experimentell bestimmt.
